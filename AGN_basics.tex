\documentclass[11pt]{article}

%\setlength {\textwidth}{180mm} 
%\setlength {\textheight}{260mm}
%\topmargin=-35.00mm
%\oddsidemargin=-10.00mm
%\pagestyle{empty}

\usepackage{graphicx,fancyhdr,natbib,subfigure}
\usepackage{amsmath, cancel, amssymb}
\usepackage{hyperref}
\usepackage{lscape}



%%%%%%%%%%%%%%%%%%%%%%%%%%%%%%%%%%%%%%%%%%%
%       define Journal abbreviations      %
%%%%%%%%%%%%%%%%%%%%%%%%%%%%%%%%%%%%%%%%%%%
\def\nat{Nat} \def\apjl{ApJ~Lett.} \def\apj{ApJ}
\def\apjs{ApJS} \def\aj{AJ} \def\mnras{MNRAS}
\def\prd{Phys.~Rev.~D} \def\prl{Phys.~Rev.~Lett.}
\def\plb{Phys.~Lett.~B} \def\jhep{JHEP}
\def\npbps{NUC.~Phys.~B~Proc.~Suppl.} \def\prep{Phys.~Rep.}
\def\pasp{PASP} \def\aap{Astron.~\&~Astrophys.} \def\araa{ARA\&A}
\newcommand{\preep}[1]{{\tt #1} }

%%%%%%%%%%%%%%%%%%%%%%%%%%%%%%%%%%%%%%%%%%%%%%%%%%%%%
%              define symbols                       %
%%%%%%%%%%%%%%%%%%%%%%%%%%%%%%%%%%%%%%%%%%%%%%%%%%%%%
\def \Mpc {~{\rm Mpc} }
\def \Om {\Omega_0}
\def \Omb {\Omega_{\rm b}}
\def \Omcdm {\Omega_{\rm CDM}}
\def \Omlam {\Omega_{\Lambda}}
\def \Omm {\Omega_{\rm m}}
\def \ho {H_0}
\def \qo {q_0}
\def \lo {\lambda_0}
\def \kms {{\rm ~km~s}^{-1}}
\def \kmsmpc {{\rm ~km~s}^{-1}~{\rm Mpc}^{-1}}
\def \hmpc{~\;h^{-1}~{\rm Mpc}} 
\def \hkpc{\;h^{-1}{\rm kpc}} 
\def \hmpcb{h^{-1}{\rm Mpc}}
\def \dif {{\rm d}}
\def \mlim {m_{\rm l}}
\def \bj {b_{\rm J}}
\def \mb {M_{\rm b_{\rm J}}}
\def \qso {_{\rm QSO}}
\def \lrg {_{\rm LRG}}
\def \gal {_{\rm gal}}
\def \xibar {\bar{\xi}}
\def \xis{\xi(s)}
\def \xisp{\xi(\sigma, \pi)}
\def \Xisig{\Xi(\sigma)}
\def \xir{\xi(r)}
\def \max {_{\rm max}}
\def \gsim { \lower .75ex \hbox{$\sim$} \llap{\raise .27ex \hbox{$>$}} }
\def \lsim { \lower .75ex \hbox{$\sim$} \llap{\raise .27ex \hbox{$<$}} }
\def \deg {^{\circ}}
\def \deltac {\delta_{\rm c}}
\def \mmin {M_{\rm min}}
\def \mbh  {M_{\rm BH}}
\def \mdh  {M_{\rm DH}}
\def \msun {M_{\odot}}
\def \z {_{\rm z}}
\def \edd {_{\rm Edd}}
\def \lin {_{\rm lin}}
\def \nonlin {_{\rm non-lin}}
\def \wrms {\langle w_{\rm z}^2\rangle^{1/2}}
\def \dc {\delta_{\rm c}}
\def \wp {w_{p}(\sigma)}
\def \PwrSp {\mathcal{P}(k)}
\def \DelSq {$\Delta^{2}(k)$}
\def \WMAP {{\it WMAP \,}}
\def \cobe {{\it COBE }}
\def \COBE {{\it COBE \;}}
\def \HST  {{\it HST \,\,}}
\def \Spitzer  {{\it Spitzer \,}}


\begin{document}


\title{Just a few really basic notes on AGN: \\
From discussions with Andy L. and David H.}
%\author{Nicholas P. Ross}
\date{\today}
\maketitle

%%

%\begin{abstract}
%This is a simple document which will... 
%\end{abstract}



\section{Where's the energy coming from??}
Note, very bright radio sources have flux densities of $\sim$a
Jansky. An iPhone 6 has an RF output of $\approx$30 dBm (at 824.2 -
848.8MHz and 1850.2 - 1909.8MHz) where a dBm is a decibel-milliwatt
and 30dBm is 1.0 Watt\footnote{https://en.wikipedia.org/wiki/DBm}. So
placing an iPhone on the Moon would give it a radio flux of...??

\noindent
The Schwarzschild radius is:
\begin{equation}
     R_{\rm Sch} = \frac{2GM}{c^2}
\end{equation}
The Event Horizon is at 2 Schwarzschild radii!!!

\noindent
How much of the energy is coming from various Schwarzschild radii??
3-5 or 3-10 $R_{\rm Sch}$ vs. 5-$\infty$ $R_{\rm Sch}$??\\

\noindent
$L = 4\pi \sigma  R^2 T^4$, so...
\begin{equation}
T = (L c^4) / \sigma \pi G^{2} 
\end{equation}

\noindent
Working this all through...
\begin{equation}
  L = 1.38\times10^{31} \, {\rm Watts} \, (M / M_{\odot})
\end{equation}

\noindent
which with  $M_{\odot} = 2 \times 10^{30}$ kg is just 
\begin{equation}
  L = 6.37 M. 
\end{equation}
in S.I. units. 
%% WTF!!! ;-)

\begin{equation}
  \Delta E_{p} = \frac{G M m}{R} 
\end{equation}
with $R=2GM/c^2$ gives
\begin{equation}
    \Delta E_{p} = \frac{mc^2}{2}
\end{equation}
but ``of course'' this wont go all into 'shining', turns into K.E., 
and you need some friction... etc. etc. etc. :-) 
But then divide by two for rotation and divide by 3 for LSO (last stable orbit). 
ie. $\sim 1/12$ . 

\begin{equation}
 E = \mu \Delta m c^{2} 
\end{equation}
and thus
\begin{equation}
 L = \mu \Delta \dot{m} c^{2} 
\end{equation}
$\mu$ is 0.7\% for nuclear fusion. \\
$\mu$ is 10-40\% for grav. potential accretion.  \\
And, $L$ for BHs doesn't depend on the mass of the BH (!!) \\

\noindent
$L$ increases with \.m, upto the Eddington Luminosity. \\
The luminosity itself cuts off itself the growth (in L). \\

\noindent
Use Wien's Displacement Law such that:\\
$\lambda$  T = 2900 ($L$ in $\mu$m and T in K)\\
$\Rightarrow$ if T = 100,000K then $\lambda=0.03\mu$m, \\
i.e. and the EUV. \\
(Note, a photon with 12.6 eV of energy has a wavelenght of 100nm.)
%% https://en.wikipedia.org/wiki/Electronvolt
So, you can expect the peak to be at $\sim300$\AA. 

\noindent
However, we see the turnover at $\sim1000$\AA, 
a little cooler than we'd expect... Why??!!



\section{Resources}
Classic References:\\
\citet{SS73} (and \citet{King09}) \\
\citet{Pringle81}\\
(also e.g., \citet{Pringle72, Pringle73, Pringle96})\\
\citet{Richards06b}\\
\citet{Kishimoto08}\\
\citet{Lawrence12}, and the paper trail therein...\\

\noindent
Good links:\\
Schwarzschild radius: https://en.wikipedia.org/wiki/Schwarzschild\_radius\\
www-astro.physics.ox.ac.uk/$\sim$garret/teaching/lecture7-2012.pdf\\
jila.colorado.edu/~pja/astr3730/lecture18.pdf\\
https://andyxl.wordpress.com/2011/03/03/a-dim-glimmer/\\



\bibliographystyle{mn2e}
\bibliography{/cos_pc19a_npr/LaTeX/tester_mnras}

\end{document}

