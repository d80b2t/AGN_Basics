\documentclass[11pt]{article}

%\setlength {\textwidth}{180mm} 
%\setlength {\textheight}{260mm}
%\topmargin=-35.00mm
%\oddsidemargin=-10.00mm
%\pagestyle{empty}

\input{format}

\begin{document}


\title{Just a few (really) basic notes on AGN: \\
From discussions with Andy L and David H.:}
\author{Nicholas P. Ross}
\date{\today}
\maketitle

%%

%\begin{abstract}
%This is a simple document which will... 
%\end{abstract}



\section{Accretion Disk Basics}
Classic References:\\
\citet{SS73}\\
\citet{Pringle81}\\
\citet{Pringle81}\\
\citet{Richards06b}\\
\citet{Kishimoto08}\\
\citet{Lawrence12}, and the paper trail therein...\\

Good links:
Schwarzschild radius: https://en.wikipedia.org/wiki/Schwarzschild\_radius
www-astro.physics.ox.ac.uk/$\sim$garret/teaching/lecture7-2012.pdf
jila.colorado.edu/~pja/astr3730/lecture18.pdf

\begin{equation}
     R_{\rm Sch} = \frac{2GM}{c^2}
\end{equation}
The Event Horizon is at 2 Schwarzschild radii !!!

$L = 4\pi \sigma  R^2 T^4$. 
\begin{equation}
T = (L c^4) / \sigma \pi G^{2} 
\end{equation}


\begin{equation}
  L = 6.37 M. 
\end{equation}
WTF. 

\begin{equation}
  \Delta E_{p} = \frac{G M m}{R} 
\end{equation}
with $R=2GM/c^2$ gives
\begin{equation}
    \Delta E_{p} = \frac{mc^2}{2}
\end{equation}
but ``of course'' this wont go all into 'shining', turns into K.E., 
and you need some friction... etc. etc. etc. :-) 
But then divide by two for rotation and divide by 3 for LSO (last stable orbit). 
ie. $\sim 1/12$ . 

\begin{equation}
 E = \mu \Delta m c^{2} 
\end{equation}
and thus
\begin{equation}
 L = \mu \Delta \dot{m} c^{2} 
\end{equation}
$\mu$ is 0.7\% for nuclear fusion. \\
$\mu$ is 10-40\% for grav. potential accretion.  \\
And, $L$ for BHs doesn't depend on the mass of the BH (!!) \\

$L$ increases with \.m, upto the Eddington Luminosity. \\
The luminosity itself cuts off itself the growth (in L). \\

$\lambda$  T = 2900 ($L$ in $\mu$m and T in K)\\
$\Rightarrow$ if T = 100,000K then $\lambda=0.03\mu$m, \\
i.e. and the EUV. \\
12.6 eV is $\sim$1 keV
Expect peak at $\sim300$\AA

However, we see the turnover at $\sim1000$\AA, 
a little cooler than we'd expect... Why??!!


%\begin{itemize}
%\item 
%\end{itemize} 



\section{References}
\citet{Ross07}

\bibliographystyle{mn2e}
\bibliography{/cos_pc19a_npr/LaTeX/tester_mnras}

\end{document}

